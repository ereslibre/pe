\documentclass[12pt]{article}
\usepackage[utf8]{inputenc}
\usepackage[spanish]{babel}
\usepackage{graphicx}
\usepackage{float}
\title{Práctica 1\\Programación Evolutiva}
\author{Rafael Fernández López\\Ángel Valero Picazo}
\date{}

\pdfinfo
{
  /Title       (PRACTICA1-PE)
  /Author      (RAFAEL FERNANDEZ LOPEZ, ANGEL VALERO PICAZO)
}

\begin{document}

\maketitle
\newpage
\newpage
\tableofcontents
\newpage

\section{Estudio Obligatorio de las Funciones}	
	Para el estudio de las 5 funciones de la práctica se han fijado los siguientes valores:
	\begin{itemize}
		\item Tamaño de población : 100
		\item Número Máximo de Generaciones : 100
		\item Probabilidad de Cruce : 0.6
		\item Probabilidad de Mutación : 0.1
		\item Precisión : 0.0000001
	\end{itemize}

\subsection{Función 1}
	La primera función presenta un máximo de 1.98442121 en 0.98449933 como se puede observar en la figura.
\begin{figure}[H]
\centering
\includegraphics[scale=0.4]{graficas/F1inicial}
\caption{Función 1.}
\label{fig}
\end{figure}
	A continuación se muestran las diferentes gráficas para observar la evolución que sufre la función.

\subsubsection*{Gráfica Evaluación}
\begin{figure}[H]
\centering
\includegraphics[scale=0.4]{graficas/F1inicial_algoritmo}
\caption{Gráfica evaluación.}
\label{fig}
\end{figure}
	Se observa los diferentes picos de cada generación y la cota superior marca el mejor de todas las generaciones.

\subsubsection*{Gráfica Aptitud}
\begin{figure}[H]
\centering
\includegraphics[scale=0.4]{graficas/F1inicial_aptitud}
\caption{Gráfica aptitud.}
\label{fig}
\end{figure}
	En esta figura se puede ver como el valor medio de aptitud va convergiendo hacia el valor máximo de aptitud.

\subsubsection*{Gráfica Presión Selectiva}
\begin{figure}[H]
\centering
\includegraphics[scale=0.4]{graficas/F1inicial_presion}
\caption{Gráfica presión selectiva.}
\label{fig}
\end{figure}
	En la selección por ruleta, si no se aplica ninguna función de escala, aplica una presión selectiva muy alta cuando las aptitudes de los individuos son variadas, y muy baja cuando las aptitudes son similares.
\newpage

\subsection{Función 2}
	La segunda función presenta un máximo de 38.330 en 11.607 y 5.526 como se puede observar en la figura.
\begin{figure}[H]
\centering
\includegraphics[scale=0.4]{graficas/F2inicial}
\caption{Función 2.}
\label{fig}
\end{figure}
	A continuación se muestran las diferentes gráficas para observar la evolución que sufre la función.

\subsubsection*{Gráfica Evaluación}
\begin{figure}[H]
\centering
\includegraphics[scale=0.4]{graficas/F2inicial_algoritmo}
\caption{Gráfica evaluación.}
\label{fig}
\end{figure}
	Se observa los diferentes picos de cada generación y la cota superior marca el mejor de todas las generaciones.

\subsubsection*{Gráfica Aptitud}
\begin{figure}[H]
\centering
\includegraphics[scale=0.4]{graficas/F2inicial_aptitud}
\caption{Gráfica aptitud}
\label{fig}
\end{figure}
	En esta figura se puede ver como el valor medio de aptitud va convergiendo hacia el valor máximo de aptitud.

\subsubsection*{Gráfica Presión Selectiva}
\begin{figure}[H]
\centering
\includegraphics[scale=0.4]{graficas/F2inicial_presion}
\caption{Gráfica presión selectiva}
\label{fig}
\end{figure}
	En la selección por ruleta, si no se aplica ninguna función de escala, aplica una presión selectiva muy alta cuando las aptitudes de los individuos son variadas, y muy baja cuando las aptitudes son similares.
\newpage

\subsection{Función 3}
	La tercera función presenta un mínimo de -0.318067 en 4.574724 como se puede observar en la figura.
\begin{figure}[H]
\centering
\includegraphics[scale=0.4]{graficas/F3inicial}
\caption{Función 3.}
\label{fig}
\end{figure}
	A continuación se muestran las diferentes gráficas para observar la evolución que sufre la función.

\subsubsection*{Gráfica Evaluación}
\begin{figure}[H]
\centering
\includegraphics[scale=0.4]{graficas/F3inicial_algoritmo}
\caption{Gráfica evaluación.}
\label{fig}
\end{figure}
	Se observa los diferentes picos de cada generación y la cota inferior marca el mejor de todas las generaciones.

\subsubsection*{Gráfica Aptitud}
\begin{figure}[H]
\centering
\includegraphics[scale=0.4]{graficas/F3inicial_aptitud}
\caption{Gráfica aptitud}
\label{fig}
\end{figure}
	En esta figura se puede ver como el valor medio de aptitud va convergiendo hacia el valor máximo de aptitud. Cabe destacar que los valores en la gráfica salen desplazados ya que se convierte un problema de minimización en otro de maximización.

\subsubsection*{Gráfica Presión Selectiva}
\begin{figure}[H]
\centering
\includegraphics[scale=0.4]{graficas/F3inicial_presion}
\caption{Gráfica presión selectiva}
\label{fig}
\end{figure}
	En la selección por ruleta, si no se aplica ninguna función de escala, aplica una presión selectiva muy alta cuando las aptitudes de los individuos son variadas, y muy baja cuando las aptitudes son similares.
\newpage

\subsection{Función 4}
	La cuarta función presenta mínimos de -184.8932 en -0.7714 y -7.71 como se puede observar en la figura.
\begin{figure}[H]
\centering
\includegraphics[scale=0.4]{graficas/F4inicial}
\caption{Función 4.}
\label{fig}
\end{figure}
	A continuación se muestran las diferentes gráficas para observar la evolución que sufre la función.

\subsubsection*{Gráfica Evaluación}
\begin{figure}[H]
\centering
\includegraphics[scale=0.4]{graficas/F4inicial_algoritmo}
\caption{Gráfica evaluación.}
\label{fig}
\end{figure}
	Se observa los diferentes picos de cada generación y la cota inferior marca el mejor de todas las generaciones.

\subsubsection*{Gráfica Aptitud}
\begin{figure}[H]
\centering
\includegraphics[scale=0.4]{graficas/F4inicial_aptitud}
\caption{Gráfica aptitud}
\label{fig}
\end{figure}
	En esta figura se puede ver como el valor medio de aptitud va convergiendo hacia el valor máximo de aptitud. Cabe destacar que los valores en la gráfica salen desplazados ya que se convierte un problema de minimización en otro de maximización.

\subsubsection*{Gráfica Presión Selectiva}
\begin{figure}[H]
\centering
\includegraphics[scale=0.4]{graficas/F4inicial_presion}
\caption{Gráfica presión selectiva}
\label{fig}
\end{figure}
	En la selección por ruleta, si no se aplica ninguna función de escala, aplica una presión selectiva muy alta cuando las aptitudes de los individuos son variadas, y muy baja cuando las aptitudes son similares. Se observa que existen algunas pequeñas subidas debido a la variedad de las aptitudes.
\newpage

\subsection{Función 5}
	La quinta función presenta un mínimos de -3.635715 cuando n=4 como se puede observar en la figura.
\begin{figure}[H]
\centering
\includegraphics[scale=0.4]{graficas/F5inicial}
\caption{Función 5.}
\label{fig}
\end{figure}
	A continuación se muestran las diferentes gráficas para observar la evolución que sufre la función.

\subsubsection*{Gráfica Evaluación}
\begin{figure}[H]
\centering
\includegraphics[scale=0.4]{graficas/F5inicial_algoritmo}
\caption{Gráfica evaluación.}
\label{fig}
\end{figure}
	Se observa los diferentes picos de cada generación y la cota inferior marca el mejor de todas las generaciones.

\subsubsection*{Gráfica Aptitud}
\begin{figure}[H]
\centering
\includegraphics[scale=0.4]{graficas/F5inicial_aptitud}
\caption{Gráfica aptitud}
\label{fig}
\end{figure}
	En esta figura se puede ver como el valor medio de aptitud va convergiendo hacia el valor máximo de aptitud. Cabe destacar que los valores en la gráfica salen desplazados ya que se convierte un problema de minimización en otro de maximización.

\subsubsection*{Gráfica Presión Selectiva}
\begin{figure}[H]
\centering
\includegraphics[scale=0.4]{graficas/F5inicial_presion}
\caption{Gráfica presión selectiva}
\label{fig}
\end{figure}
	En la selección por ruleta, si no se aplica ninguna función de escala, aplica una presión selectiva muy alta cuando las aptitudes de los individuos son variadas, y muy baja cuando las aptitudes son similares. Se observa que existen algunas pequeñas subidas debido a la variedad de las aptitudes.	
\newpage

\section{Estudio Opcional de las Funciones}	
	Para el estudio de las 5 funciones de la práctica se han fijado los siguientes valores:
	\begin{itemize}
		\item Tamaño de población : 100
		\item Número Máximo de Generaciones : 100
		\item Probabilidad de Cruce : 0.6
		\item Probabilidad de Mutación : 0.1
		\item Precisión : 0.0000001
	\end{itemize}
	A la hora de estudiar la influencia de un cierto parámetro hay que indicar el rango de valores y el incremento de cada iteración. Los demás parámetros se quedan fijados con los valores anteriores.
\subsection{Función 1}
\subsubsection*{Estudio Tamaño de Población}
	Se varia el tamaño de la población de 10 a 460 en incrementos de 50.
%tabla
\begin{table}[H]
\begin{center}
\begin{tabular}{|cc|} \hline
Tamaño Población & Máximo Obtenido \\  \hline
10  & 1.95300216 \\ 
60  & 1.97934328 \\ 
110 & 1.98414916 \\
160 & 1.98371034 \\
210 & 1.98417869 \\
260 & 1.98442445 \\
310 & 1.98441514 \\
360 & 1.98442447 \\ 
410 & 1.98441983 \\
460 & 1.98442446 \\  \hline
\end{tabular}
\end{center}
\end{table}
	Los valores de tamaño de población que hacen que la función presente un buen máximo están de 260 a 460. Es lógico ya que contra mayor sea la población mas preciso es el algoritmo genético.   

\subsubsection*{Estudio Número Máximo de Generaciones}
	Se varia el número máximo de generaciones de 10 a 460 en incrementos de 50.
%tabla
\begin{table}[H]
\begin{center}
\begin{tabular}{|cc|} \hline
Número Máximo Generaciones & Máximo Obtenido \\  \hline
10  & 1.98420692 \\ 
60  & 1.98371961 \\ 
110 & 1.98428048 \\
160 & 1.98440826 \\
210 & 1.98428711 \\
260 & 1.98312992 \\
310 & 1.98398901 \\
360 & 1.98415210 \\ 
410 & 1.98441996 \\
460 & 1.98437203 \\  \hline
\end{tabular}
\end{center}
\end{table}
	Se observa que en con esta función varia mínimamente el mejor, parece que entorno a 100-160 es un buen rango para obtener una buena medida. Al aumentar mucho el número de generaciones no se consiguen grandes mejoras.

\subsubsection*{Estudio Probabilidad de Cruce}
	Se varia la probabilidad de cruce de 0.1 a 1 en incrementos de 0.1.
%tabla
\begin{table}[H]
\begin{center}
\begin{tabular}{|cc|} \hline
Probabilidad Cruce & Máximo Obtenido \\  \hline
0.1 & 1.96842359 \\ 
0.2 & 1.98435401 \\ 
0.3 & 1.98146319 \\
0.4 & 1.95309093 \\
0.5 & 1.98430322 \\
0.6 & 1.95271094 \\
0.7 & 1.98437476 \\
0.8 & 1.98441772 \\ 
0.9 & 1.98364390 \\
1   & 1.98392662 \\  \hline
\end{tabular}
\end{center}
\end{table}
	Un buen valor de probabilidad de cruce se sitúa entorno a 0.7-0.8.

\subsubsection*{Estudio Probabilidad de Mutación}
	Se varia la probabilidad de mutación de 0.1 a 1 en incrementos de 0.1.
%tabla
\begin{table}[H]
\begin{center}
\begin{tabular}{|cc|} \hline
Probabilidad Mutación & Máximo Obtenido \\  \hline
0.1 & 1.98437036 \\ 
0.2 & 1.98427642 \\ 
0.3 & 1.98003732 \\
0.4 & 1.95247656 \\
0.5 & 1.98440824 \\
0.6 & 1.98440908 \\
0.7 & 1.98437398 \\
0.8 & 1.98434554 \\ 
0.9 & 1.98274766 \\
1   & 1.98261901 \\  \hline
\end{tabular}
\end{center}
\end{table}
	En este caso la mejor probabilidad de mutación se sitúa entorno a 0.1-0.2 y entre 0.5-0.6. Al aumentar va empeorando el resultado.	

\subsubsection*{Estudio Precisión}
	Se varia la precisión de 0.00000001 a 0.0000001 en incrementos de 0.00000001.
%tabla
\begin{table}[H]
\begin{center}
\begin{tabular}{|cc|} \hline
Precisión & Máximo Obtenido \\  \hline
0.00000001 & 1.98439835 \\ 
0.00000002 & 1.98436467 \\ 
0.00000003 & 1.98381407 \\
0.00000004 & 1.98425537 \\
0.00000005 & 1.98424478 \\
0.00000006 & 1.98437295 \\
0.00000007 & 1.98398618 \\
0.00000008 & 1.98441399 \\ 
0.00000009 & 1.98442388 \\
0.00000010 & 1.98442446 \\  \hline
\end{tabular}
\end{center}
\end{table}
	Respecto a la precisión al aumentar va mejorando aunque no existe una gran diferencia.

\subsection{Función 2}
\subsubsection*{Estudio Tamaño de Población}
	Se varia el tamaño de la población de 10 a 460 en incrementos de 50.
%tabla
\begin{table}[H]
\begin{center}
\begin{tabular}{|cc|} \hline
Tamaño Población & Máximo Obtenido \\  \hline
10  & 27.814 \\ 
60  & 36.559 \\ 
110 & 36.542 \\
160 & 38.349 \\
210 & 37.983 \\
260 & 38.198 \\
310 & 38.709 \\
360 & 38.444 \\ 
410 & 38.557 \\
460 & 38.825 \\  \hline
\end{tabular}
\end{center}
\end{table}
	El valor de tamaño de población óptimo para esta función se encuentra entorno 110-160, cabe destacar que si aumentamos considerablemente el tamaño mejora levemente pero se pierde eficiencia.

\subsubsection*{Estudio Número Máximo de Generaciones}
	Se varia el número máximo de generaciones de 10 a 460 en incrementos de 50.
%tabla
\begin{table}[H]
\begin{center}
\begin{tabular}{|cc|} \hline
Número Máximo Generaciones & Máximo Obtenido \\  \hline
10  & 36.809 \\ 
60  & 38.107 \\ 
110 & 37.408 \\
160 & 37.132 \\
210 & 38.525 \\
260 & 37.483 \\
310 & 37.196 \\
360 & 38.595 \\ 
410 & 37.645 \\
460 & 37.806 \\  \hline
\end{tabular}
\end{center}
\end{table}
	EL valor óptimo del número máximo de generaciones para esta función se sitúa entorno a 210. Si se aumenta dicho valor se produce una leve mejora entorno a 360, pero para ello se empeora el coste en tiempo del algoritmo.

\subsubsection*{Estudio Probabilidad de Cruce}
	Se varia la probabilidad de cruce de 0.1 a 1 en incrementos de 0.1.
%tabla
\begin{table}[H]
\begin{center}
\begin{tabular}{|cc|} \hline
Probabilidad Cruce & Máximo Obtenido \\  \hline
0.1 & 37.629 \\ 
0.2 & 38.146 \\ 
0.3 & 36.738 \\
0.4 & 37.894 \\
0.5 & 38.746 \\
0.6 & 37.302 \\
0.7 & 36.948 \\
0.8 & 38.793 \\ 
0.9 & 37.733 \\
1   & 38.683 \\  \hline
\end{tabular}
\end{center}
\end{table}
	El valor de la probabilidad de cruce que mas ajusta el máximo esta entorno a 0.8.

\subsubsection*{Estudio Probabilidad de Mutación}
	Se varia la probabilidad de mutación de 0.1 a 1 en incrementos de 0.1.
%tabla
\begin{table}[H]
\begin{center}
\begin{tabular}{|cc|} \hline
Probabilidad Mutación & Máximo Obtenido \\  \hline
0.1 & 38.091 \\ 
0.2 & 38.719 \\ 
0.3 & 38.039 \\
0.4 & 37.945 \\
0.5 & 38.532 \\
0.6 & 38.450 \\
0.7 & 37.630 \\
0.8 & 37.748 \\ 
0.9 & 38.091 \\
1   & 36.527 \\  \hline
\end{tabular}
\end{center}
\end{table}
	Respecto a la probabilidad de mutación el mejor valor que se puede observar esta entorno a 0.2.

\subsubsection*{Estudio Precisión}
	Se varia la precisión de 0.00000001 a 0.0000001 en incrementos de 0.00000001.
%tabla
\begin{table}[H]
\begin{center}
\begin{tabular}{|cc|} \hline
Precisión & Máximo Obtenido \\  \hline
0.00000001 & 36.722 \\ 
0.00000002 & 38.447 \\ 
0.00000003 & 35.648 \\
0.00000004 & 37.240 \\
0.00000005 & 38.647 \\
0.00000006 & 38.323 \\
0.00000007 & 38.451\\
0.00000008 & 37.847 \\ 
0.00000009 & 37.600 \\
0.00000010 & 37.837 \\  \hline
\end{tabular}
\end{center}
\end{table}
	La precisión influye poco pero se puede destacar que entorno a 0.00000005 se comporta ligeramente mejor. 

\subsection{Función 3}
\subsubsection*{Estudio Tamaño de Población}
	Se varia el tamaño de la población de 10 a 460 en incrementos de 50.
%tabla
\begin{table}[H]
\begin{center}
\begin{tabular}{|cc|} \hline
Tamaño Población & Mínimo Obtenido \\  \hline
10  & -0.269028 \\ 
60  & -0.317650 \\ 
110 & -0.316676 \\
160 & -0.318071 \\
210 & -0.318069 \\
260 & -0.318071 \\
310 & -0.318071 \\
360 & -0.318071 \\ 
410 & -0.318071 \\
460 & -0.318071 \\  \hline
\end{tabular}
\end{center}
\end{table}
	En este estudio se observa claramente que el tamaño de población a partir de 260 se comporta de manera óptima. 

\subsubsection*{Estudio Número Máximo de Generaciones}
	Se varia el número máximo de generaciones de 10 a 460 en incrementos de 50.
%tabla
\begin{table}[H]
\begin{center}
\begin{tabular}{|cc|} \hline
Número Máximo Generaciones & Mínimo Obtenido \\  \hline
10  & -0.318035 \\ 
60  & -0.318071 \\ 
110 & -0.318071 \\
160 & -0.318061 \\
210 & -0.318071 \\
260 & -0.316046 \\
310 & -0.316980 \\
360 & -0.318070 \\ 
410 & -0.318071 \\
460 & -0.318038 \\  \hline
\end{tabular}
\end{center}
\end{table}
	De 60-160 es un buen intervalo para el número máximo de generaciones. Si se sigue aumentando se llega también a buenos valores pero el rendimiento del algoritmo empeora.

\subsubsection*{Estudio Probabilidad de Cruce}
	Se varia la probabilidad de cruce de 0.1 a 1 en incrementos de 0.1.
%tabla
\begin{table}[H]
\begin{center}
\begin{tabular}{|cc|} \hline
Probabilidad Cruce & Mínimo Obtenido \\  \hline
0.1 & -0.317957 \\ 
0.2 & -0.315544 \\ 
0.3 & -0.318038 \\
0.4 & -0.317997 \\
0.5 & -0.318056 \\
0.6 & -0.318048 \\
0.7 & -0.318071 \\
0.8 & -0.318071 \\ 
0.9 & -0.318069 \\
1   & -0.315809 \\  \hline
\end{tabular}
\end{center}
\end{table}
	Entorno a 0.7-0.8 es una buena aproximación para la probabilidad de cruce.	

\subsubsection*{Estudio Probabilidad de Mutación}
	Se varia la probabilidad de mutación de 0.1 a 1 en incrementos de 0.1.
%tabla
\begin{table}[H]
\begin{center}
\begin{tabular}{|cc|} \hline
Probabilidad Mutación & Mínimo Obtenido \\  \hline
0.1 & -0.318067 \\ 
0.2 & -0.317451 \\ 
0.3 & -0.318071 \\
0.4 & -0.318070 \\
0.5 & -0.318051 \\
0.6 & -0.318068 \\
0.7 & -0.318054 \\
0.8 & -0.318071 \\ 
0.9 & -0.318031 \\
1   & -0.318071 \\  \hline
\end{tabular}
\end{center}
\end{table}
	En la probabilidad de mutación esta muy igualada, se puede destacar ligeramente entorno a 0.3.

\subsubsection*{Estudio Precisión}
	Se varia la precisión de 0.00000001 a 0.0000001 en incrementos de 0.00000001.
%tabla
\begin{table}[H]
\begin{center}
\begin{tabular}{|cc|} \hline
Precisión & Mínimo Obtenido \\  \hline
0.00000001 & -0.318061 \\ 
0.00000002 & -0.318043 \\ 
0.00000003 & -0.318070 \\
0.00000004 & -0.318069 \\
0.00000005 & -0.317839 \\
0.00000006 & -0.316134 \\
0.00000007 & -0.317775 \\
0.00000008 & -0.317918 \\ 
0.00000009 & -0.318071 \\
0.00000010 & -0.318071 \\  \hline
\end{tabular}
\end{center}
\end{table}
	La precisión esta muy igualada y no hay grandes diferencias pero se puede observar que mejora ligeramente entorno a 0.00000009.

\subsection{Función 4}
\subsubsection*{Estudio Tamaño de Población}
	Se varia el tamaño de la población de 10 a 460 en incrementos de 50.
%tabla
\begin{table}[H]
\begin{center}
\begin{tabular}{|cc|} \hline
Tamaño Población & Mínimo Obtenido \\  \hline
10  & -72.9491 \\ 
60  & -111.6176 \\ 
110 & -135.3392 \\
160 & -164.1953 \\
210 & -186.7028 \\
260 & -122.3201 \\
310 & -186.6605 \\
360 & -186.4939 \\ 
410 & -186.6807 \\
460 & -183.2034 \\  \hline
\end{tabular}
\end{center}
\end{table}
	En esta función los valores mas adecuados para hallar el mínimo se encuentran en torno 260-410, al aumentar el tamaño de la población se mejoran los resultados notablemente. 

\subsubsection*{Estudio Número Máximo de Generaciones}
	Se varia el número máximo de generaciones de 10 a 460 en incrementos de 50.
%tabla
\begin{table}[H]
\begin{center}
\begin{tabular}{|cc|} \hline
Número Máximo Generaciones & Mínimo Obtenido \\  \hline
10  & -184.9346 \\ 
60  & -92.6084 \\ 
110 & -130.5393 \\
160 & -181.9670 \\
210 & -186.0169 \\
260 & -186.1944 \\
310 & -181.9110 \\
360 & -172.6664 \\ 
410 & -180.1084 \\
460 & -183.5149 \\  \hline
\end{tabular}
\end{center}
\end{table}
	Los valores más óptimos para encontrar el mínimo de esta función se encuentran en el rango 210-260. Si se sale de dicho rango el resultado parece empeorar.

\subsubsection*{Estudio Probabilidad de Cruce}
	Se varia la probabilidad de cruce de 0.1 a 1 en incrementos de 0.1.
%tabla
\begin{table}[H]
\begin{center}
\begin{tabular}{|cc|} \hline
Probabilidad Cruce & Mínimo Obtenido \\  \hline
0.1 & -183.4316 \\ 
0.2 & -185.9303 \\ 
0.3 & -129.8455 \\
0.4 & -173.6409 \\
0.5 & -78.5388 \\
0.6 & -167.1205 \\
0.7 & -146.5928 \\
0.8 & -185.6488 \\ 
0.9 & -185.7041 \\
1   & -185.5169 \\  \hline
\end{tabular}
\end{center} 
\end{table}
	En el estudio los mejores resultados para para la probabilidad de cruce se sitúan entorno a 0.8-0.9. Si se dan valores muy bajos empeora notablemente el resultado.

\subsubsection*{Estudio Probabilidad de Mutación}
	Se varia la probabilidad de mutación de 0.1 a 1 en incrementos de 0.1.
%tabla
\begin{table}[H]
\begin{center}
\begin{tabular}{|cc|} \hline
Probabilidad Mutación & Mínimo Obtenido \\  \hline
0.1 & -181.7142 \\ 
0.2 & -186.3339 \\ 
0.3 & -184.8755 \\
0.4 & -168.8033 \\
0.5 & -176.6376 \\
0.6 & -186.1045 \\
0.7 & -186.0351 \\
0.8 & -130.8406 \\ 
0.9 & -176.9818 \\
1   & -117.5346 \\  \hline
\end{tabular}
\end{center}
\end{table}
	Para la probabilidad de mutación en esta función los mejores resultados se dan con una probabilidad cercana a 0.2.

\subsubsection*{Estudio Precisión}
	Se varia la precisión de 0.00000001 a 0.0000001 en incrementos de 0.00000001.
%tabla
\begin{table}[H]
\begin{center}
\begin{tabular}{|cc|} \hline
Precisión & Mínimo Obtenido \\  \hline
0.00000001 & -157.2270 \\ 
0.00000002 & -100.5084 \\ 
0.00000003 & -182.6262 \\
0.00000004 & -182.6732 \\
0.00000005 & -175.5188 \\
0.00000006 & -156.3064 \\
0.00000007 & -186.4751 \\
0.00000008 & -183.7943 \\ 
0.00000009 & -182.0350 \\
0.00000010 & -103.3778 \\  \hline
\end{tabular}
\end{center}
\end{table}
	Respecto a la precisión el intervalo donde mejor se comporta la función es 0.00000007-0.00000008.

\subsection{Función 5}
	Estudio realizado con el parámetro n=2
\subsubsection*{Estudio Tamaño de Población}
	Se varia el tamaño de la población de 10 a 460 en incrementos de 50.
%tabla
\begin{table}[H]
\begin{center}
\begin{tabular}{|cc|} \hline
Tamaño Población & Mínimo Obtenido \\  \hline
10  & -1.796137 \\ 
60  & -1.814462 \\ 
110 & -1.946951 \\
160 & -1.958608 \\
210 & -1.957552 \\
260 & -1.955446 \\
310 & -1.957697 \\
360 & -1.951744 \\ 
410 & -1.957041 \\
460 & -1.959072 \\  \hline
\end{tabular}
\end{center}
\end{table}
	El tamaño de población adecuado para esta función se corresponde con el intervalo 160-460 aunque cabe destacar que conforme se aumenta se aproxima más al mínimo buscado.	 

\subsubsection*{Estudio Número Máximo de Generaciones}
	Se varia el número máximo de generaciones de 10 a 460 en incrementos de 50.
%tabla
\begin{table}[H]
\begin{center}
\begin{tabular}{|cc|} \hline
Número Máximo Generaciones & Mínimo Obtenido \\  \hline
10  & -1.881701 \\ 
60  & -1.956846 \\ 
110 & -1.952919 \\
160 & -1.955062 \\
210 & -1.955691 \\
260 & -1.913593 \\
310 & -1.942198 \\
360 & -1.957594 \\ 
410 & -1.928775 \\
460 & -1.939548 \\  \hline
\end{tabular}
\end{center}
\end{table}
	Entorno al intervalo 60-210 se encuentra el mejor valor para el número máximo de generaciones. Si se sigue aumentando se pierde precisión respecto al mínimo.

\subsubsection*{Estudio Probabilidad de Cruce}
	Se varia la probabilidad de cruce de 0.1 a 1 en incrementos de 0.1.
%tabla
\begin{table}[H]
\begin{center}
\begin{tabular}{|cc|} \hline
Probabilidad Cruce & Mínimo Obtenido \\  \hline
0.1 & -1.812777 \\ 
0.2 & -1.959031 \\ 
0.3 & -1.954478 \\
0.4 & -1.693149 \\
0.5 & -1.862722 \\
0.6 & -1.907169 \\
0.7 & -1.916014 \\
0.8 & -1.957601 \\ 
0.9 & -1.790226 \\
1   & -1.957731 \\  \hline
\end{tabular}
\end{center}
\end{table}
	De los valores estudiados cabe destacar 0.8 como la probabilidad de cruce más adecuada para esta función.

\subsubsection*{Estudio Probabilidad de Mutación}
	Se varia la probabilidad de mutación de 0.1 a 1 en incrementos de 0.1.
%tabla
\begin{table}[H]
\begin{center}
\begin{tabular}{|cc|} \hline
Probabilidad Mutación & Mínimo Obtenido \\  \hline
0.1 & -1.958825 \\ 
0.2 & -1.955002 \\ 
0.3 & -1.954345 \\
0.4 & -1.958458 \\
0.5 & -1.784000 \\
0.6 & -1.958566 \\
0.7 & -1.936031 \\
0.8 & -1.937824 \\ 
0.9 & -1.958157 \\
1   & -1.948154 \\  \hline
\end{tabular}
\end{center}
\end{table}
	Esta función ofrece un buen resultado con una probabilidad de mutación cercana al intervalo 0.1-0.4.

\subsubsection*{Estudio Precisión}
	Se varia la precisión de 0.00000001 a 0.0000001 en incrementos de 0.00000001.
%tabla
\begin{table}[H]
\begin{center}
\begin{tabular}{|cc|} \hline
Precisión & Mínimo Obtenido \\  \hline
0.00000001 & -1.954301 \\ 
0.00000002 & -1.952432 \\ 
0.00000003 & -1.956418 \\
0.00000004 & -1.953366 \\
0.00000005 & -1.918998 \\
0.00000006 & -1.956360 \\
0.00000007 & -1.935455 \\
0.00000008 & -1.900311 \\ 
0.00000009 & -1.917352 \\
0.00000010 & -1.946798 \\  \hline
\end{tabular}
\end{center}
\end{table}
	En este estudio se observa que donde más se ajusta el resultado de la función es con una precisión cercana a 0.00000003. 

\section{Conclusiones}
	Después de un largo estudio de las cinco funciones que componen la prácticas, se puede ver que los valores teóricos vistos en clase sobre el algoritmo evolutivo se corresponden bastante bien con los valores que se han ido obteniendo en los distintos estudios realizados. Se puede destacar que conforme se aumenta el tamaño de la población el algoritmo tiende a tardar más tiempo y se mejora levemente los resultados. Otros dos parámetros destacables del algoritmo evolutivos que se suelen ajustar a los valores teóricos son los de probabilidad de cruce y probabilidad de mutación, suelen estar entre 0.6 y 0.8 el de cruce y entre 0.1 y 0.2 el de mutación.	

\end{document}
