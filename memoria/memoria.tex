\documentclass[12pt]{article}
\usepackage[utf8]{inputenc}
\usepackage[spanish]{babel}
\usepackage{graphicx}
\usepackage{float}
\title{Práctica 1\\Programación Evolutiva}
\author{Rafael Fernández López\\Ángel Valero Picazo}
\date{}

\pdfinfo
{
  /Title       (PRACTICA1-PE)
  /Author      (RAFAEL FERNANDEZ LOPEZ, ANGEL VALERO PICAZO)
}

\begin{document}

\maketitle
\newpage
\newpage
\tableofcontents
\newpage

\section{Estudio Obligatorio de las Funciones}	
	Para el estudio de las 5 funciones de la práctica se han fijado los siguientes valores:
	\begin{itemize}
		\item Tamaño de población : 100
		\item Número Máximo de Generaciones : 100
		\item Probabilidad de Cruce : 0.6
		\item Probabilidad de Mutación : 0.1
		\item Precisión : 0.0000001
	\end{itemize}

\subsection{Función 1}
	La primera función presenta un máximo de 1.98442121 en 0.98449933 como se puede observar en la figura.
\begin{figure}[H]
\centering
\includegraphics[scale=0.5]{graficas/F1inicial}
\caption{Función 1.}
\label{fig}
\end{figure}
	A continuación se muestran las diferentes gráficas para poder observar la evolución que sufre la función.

\subsubsection*{Gráfica Evaluación}
\begin{figure}[H]
\centering
\includegraphics[scale=0.5]{graficas/F1inicial_algoritmo}
\caption{Gráfica evaluación.}
\label{fig}
\end{figure}

\subsubsection*{Gráfica Aptitud}
\begin{figure}[H]
\centering
\includegraphics[scale=0.5]{graficas/F1inicial_aptitud}
\caption{Gráfica aptitud.}
\label{fig}
\end{figure}

\subsubsection*{Gráfica Presión Selectiva}
\begin{figure}[H]
\centering
\includegraphics[scale=0.5]{graficas/F1inicial_presion}
\caption{Gráfica presión selectiva.}
\label{fig}
\end{figure}
\newpage

\subsection{Función 2}
	La segunda función presenta un máximo de 38.330 en 11.607 y 5.526 como se puede observar en la figura.
\begin{figure}[H]
\centering
\includegraphics[scale=0.5]{graficas/F2inicial}
\caption{Función 2.}
\label{fig}
\end{figure}
	A continuación se muestran las diferentes gráficas para poder observar la evolución que sufre la función.

\subsubsection*{Gráfica Evaluación}
\begin{figure}[H]
\centering
\includegraphics[scale=0.5]{graficas/F2inicial_algoritmo}
\caption{Gráfica evaluación.}
\label{fig}
\end{figure}

\subsubsection*{Gráfica Aptitud}
\begin{figure}[H]
\centering
\includegraphics[scale=0.5]{graficas/F2inicial_aptitud}
\caption{Gráfica aptitud}
\label{fig}
\end{figure}

\subsubsection*{Gráfica Presión Selectiva}
\begin{figure}[H]
\centering
\includegraphics[scale=0.5]{graficas/F2inicial_presion}
\caption{Gráfica presión selectiva}
\label{fig}
\end{figure}
\newpage

\subsection{Función 3}
	La tercera función presenta un mínimo de -0.318067 en 4.574724 como se puede observar en la figura.
\begin{figure}[H]
\centering
\includegraphics[scale=0.5]{graficas/F3inicial}
\caption{Función 3.}
\label{fig}
\end{figure}
	A continuación se muestran las diferentes gráficas para poder observar la evolución que sufre la función.

\subsubsection*{Gráfica Evaluación}
\begin{figure}[H]
\centering
\includegraphics[scale=0.5]{graficas/F3inicial_algoritmo}
\caption{Gráfica evaluación.}
\label{fig}
\end{figure}

\subsubsection*{Gráfica Aptitud}
\begin{figure}[H]
\centering
\includegraphics[scale=0.5]{graficas/F3inicial_aptitud}
\caption{Gráfica aptitud}
\label{fig}
\end{figure}

\subsubsection*{Gráfica Presión Selectiva}
\begin{figure}[H]
\centering
\includegraphics[scale=0.5]{graficas/F3inicial_presion}
\caption{Gráfica presión selectiva}
\label{fig}
\end{figure}
\newpage

\subsection{Función 4}
	La cuarta función presenta mínimos en de -184.8932 en -0.7714 y -7.71 como se puede observar en la figura.
\begin{figure}[H]
\centering
\includegraphics[scale=0.5]{graficas/F4inicial}
\caption{Función 4.}
\label{fig}
\end{figure}
	A continuación se muestran las diferentes gráficas para poder observar la evolución que sufre la función.

\subsubsection*{Gráfica Evaluación}
\begin{figure}[H]
\centering
\includegraphics[scale=0.5]{graficas/F4inicial_algoritmo}
\caption{Gráfica evaluación.}
\label{fig}
\end{figure}

\subsubsection*{Gráfica Aptitud}
\begin{figure}[H]
\centering
\includegraphics[scale=0.5]{graficas/F4inicial_aptitud}
\caption{Gráfica aptitud}
\label{fig}
\end{figure}

\subsubsection*{Gráfica Presión Selectiva}
\begin{figure}[H]
\centering
\includegraphics[scale=0.5]{graficas/F4inicial_presion}
\caption{Gráfica presión selectiva}
\label{fig}
\end{figure}
\newpage

\subsection{Función 5}
	La quinta función presenta un mínimos de -3.635715 cuando n=4 como se puede observar en la figura.
\begin{figure}[H]
\centering
\includegraphics[scale=0.5]{graficas/F5inicial}
\caption{Función 5.}
\label{fig}
\end{figure}
	A continuación se muestran las diferentes gráficas para poder observar la evolución que sufre la función.

\subsubsection*{Gráfica Evaluación}
\begin{figure}[H]
\centering
\includegraphics[scale=0.5]{graficas/F5inicial_algoritmo}
\caption{Gráfica evaluación.}
\label{fig}
\end{figure}

\subsubsection*{Gráfica Aptitud}
\begin{figure}[H]
\centering
\includegraphics[scale=0.5]{graficas/F5inicial_aptitud}
\caption{Gráfica aptitud}
\label{fig}
\end{figure}

\subsubsection*{Gráfica Presión Selectiva}
\begin{figure}[H]
\centering
\includegraphics[scale=0.5]{graficas/F5inicial_presion}
\caption{Gráfica presión selectiva}
\label{fig}
\end{figure}
\newpage

\section{Estudio Opcional de las Funciones}	
	Para el estudio de las 5 funciones de la práctica se han fijado los siguientes valores:
	\begin{itemize}
		\item Tamaño de población : 100
		\item Número Máximo de Generaciones : 100
		\item Probabilidad de Cruce : 0.6
		\item Probabilidad de Mutación : 0.1
		\item Precisión : 0.0000001
	\end{itemize}
	A la hora de estudiar la influencia de un cierto parámetro hay que indicar el rango de valores y el incremento de cada iteración. Los demás parámetros se quedan fijados con los valores anteriores.
\subsection{Función 1}
\subsubsection*{Estudio Tamaño de Población}
	Se varia el tamaño de la población de 10 a 460 en incrementos de 50.
%tabla
\begin{table}[H]
\begin{center}
\begin{tabular}{|cc|} \hline
Tamaño Población & Máximo Obtenido \\  \hline
10  & 1.95300216 \\ 
60  & 1.97934328 \\ 
110 & 1.98414916 \\
160 & 1.98371034 \\
210 & 1.98417869 \\
260 & 1.98442445 \\
310 & 1.98441514 \\
360 & 1.98442447 \\ 
410 & 1.98441983 \\
460 & 1.98442446 \\  \hline
\end{tabular}
\end{center}
\end{table}
 

\subsubsection*{Estudio Número Máximo de Generaciones}
	Se varia el número máximo de generaciones de 10 a 460 en incrementos de 50.
%tabla
\begin{table}[H]
\begin{center}
\begin{tabular}{|cc|} \hline
Número Máximo Generaciones & Máximo Obtenido \\  \hline
10  & 1.98420692 \\ 
60  & 1.98371961 \\ 
110 & 1.98428048 \\
160 & 1.98440826 \\
210 & 1.98428711 \\
260 & 1.98312992 \\
310 & 1.98398901 \\
360 & 1.98415210 \\ 
410 & 1.98441996 \\
460 & 1.98437203 \\  \hline
\end{tabular}
\end{center}
\end{table}
\subsubsection*{Estudio Probabilidad de Cruce}
	Se varia la probabilidad de cruce de 0.1 a 1 en incrementos de 0.1.
%tabla
\begin{table}[H]
\begin{center}
\begin{tabular}{|cc|} \hline
Probabilidad Cruce & Máximo Obtenido \\  \hline
0.1 & 1.96842359 \\ 
0.2 & 1.98435401 \\ 
0.3 & 1.98146319 \\
0.4 & 1.95309093 \\
0.5 & 1.98430322 \\
0.6 & 1.95271094 \\
0.7 & 1.98437476 \\
0.8 & 1.98441772 \\ 
0.9 & 1.98364390 \\
1   & 1.98392662 \\  \hline
\end{tabular}
\end{center}
\end{table}
\subsubsection*{Estudio Probabilidad de Mutación}
	Se varia la probabilidad de mutación de 0.1 a 1 en incrementos de 0.1.
%tabla
\begin{table}[H]
\begin{center}
\begin{tabular}{|cc|} \hline
Probabilidad Mutación & Máximo Obtenido \\  \hline
0.1 & 1.98437036 \\ 
0.2 & 1.98427642 \\ 
0.3 & 1.98003732 \\
0.4 & 1.95247656 \\
0.5 & 1.98440824 \\
0.6 & 1.98440908 \\
0.7 & 1.98437398 \\
0.8 & 1.98434554 \\ 
0.9 & 1.98274766 \\
1   & 1.98261901 \\  \hline
\end{tabular}
\end{center}
\end{table}
\subsubsection*{Estudio Precisión}
	Se varia la precisión de 0.00000001 a 0.0000001 en incrementos de 0.00000001.
%tabla
\begin{table}[H]
\begin{center}
\begin{tabular}{|cc|} \hline
Precisión & Máximo Obtenido \\  \hline
0.00000001 & 1.98439835 \\ 
0.00000002 & 1.98436467 \\ 
0.00000003 & 1.98381407 \\
0.00000004 & 1.98425537 \\
0.00000005 & 1.98424478 \\
0.00000006 & 1.98437295 \\
0.00000007 & 1.98398618 \\
0.00000008 & 1.98441399 \\ 
0.00000009 & 1.98442388 \\
0.00000010 & 1.98442446 \\  \hline
\end{tabular}
\end{center}
\end{table}

\subsection{Función 2}
\subsubsection*{Estudio Tamaño de Población}
	Se varia el tamaño de la población de 10 a 460 en incrementos de 50.
%tabla
\begin{table}[H]
\begin{center}
\begin{tabular}{|cc|} \hline
Tamaño Población & Máximo Obtenido \\  \hline
10  & 27.814 \\ 
60  & 36.559 \\ 
110 & 36.542 \\
160 & 38.349 \\
210 & 37.983 \\
260 & 38.198 \\
310 & 38.709 \\
360 & 38.444 \\ 
410 & 38.557 \\
460 & 38.825 \\  \hline
\end{tabular}
\end{center}
\end{table}
 

\subsubsection*{Estudio Número Máximo de Generaciones}
	Se varia el número máximo de generaciones de 10 a 460 en incrementos de 50.
%tabla
\begin{table}[H]
\begin{center}
\begin{tabular}{|cc|} \hline
Número Máximo Generaciones & Máximo Obtenido \\  \hline
10  & 36.809 \\ 
60  & 38.107 \\ 
110 & 37.408 \\
160 & 37.132 \\
210 & 38.525 \\
260 & 37.483 \\
310 & 37.196 \\
360 & 38.595 \\ 
410 & 37.645 \\
460 & 37.806 \\  \hline
\end{tabular}
\end{center}
\end{table}
\subsubsection*{Estudio Probabilidad de Cruce}
	Se varia la probabilidad de cruce de 0.1 a 1 en incrementos de 0.1.
%tabla
\begin{table}[H]
\begin{center}
\begin{tabular}{|cc|} \hline
Probabilidad Cruce & Máximo Obtenido \\  \hline
0.1 & 37.629 \\ 
0.2 & 38.146 \\ 
0.3 & 36.738 \\
0.4 & 37.894 \\
0.5 & 38.746 \\
0.6 & 37.302 \\
0.7 & 36.948 \\
0.8 & 38.793 \\ 
0.9 & 37.733 \\
1   & 38.683 \\  \hline
\end{tabular}
\end{center}
\end{table}
\subsubsection*{Estudio Probabilidad de Mutación}
	Se varia la probabilidad de mutación de 0.1 a 1 en incrementos de 0.1.
%tabla
\begin{table}[H]
\begin{center}
\begin{tabular}{|cc|} \hline
Probabilidad Mutación & Máximo Obtenido \\  \hline
0.1 & 38.091 \\ 
0.2 & 38.719 \\ 
0.3 & 38.039 \\
0.4 & 37.945 \\
0.5 & 38.532 \\
0.6 & 38.450 \\
0.7 & 37.630 \\
0.8 & 37.748 \\ 
0.9 & 38.091 \\
1   & 36.527 \\  \hline
\end{tabular}
\end{center}
\end{table}
\subsubsection*{Estudio Precisión}
	Se varia la precisión de 0.00000001 a 0.0000001 en incrementos de 0.00000001.
%tabla
\begin{table}[H]
\begin{center}
\begin{tabular}{|cc|} \hline
Precisión & Máximo Obtenido \\  \hline
0.00000001 & 36.722 \\ 
0.00000002 & 38.447 \\ 
0.00000003 & 35.648 \\
0.00000004 & 37.240 \\
0.00000005 & 38.647 \\
0.00000006 & 38.323 \\
0.00000007 & 38.451\\
0.00000008 & 37.847 \\ 
0.00000009 & 37.600 \\
0.00000010 & 37.837 \\  \hline
\end{tabular}
\end{center}
\end{table}


	




\end{document}
